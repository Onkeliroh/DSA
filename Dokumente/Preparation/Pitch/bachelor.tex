\section{Grundlegende Ideen}
Die Bachelorarbeit soll zum Ziel haben ein Programm zu erarbeiten, mit dessen 
Hilfe ein informatisch unversierter Nutzer Messungen in einer
elektrotechnischen Schaltung vornehmen kann ohne vorhandene 
Programmierkenntnisse.

\section{Haupt Inhalte}
\begin{itemize}
 \item Nutzerfreundliches Interface
 \item ``einfaches'' Handhabung der konfiguration
  \subitem das schliesst das Aufspielen der Arduino-Software mit ein
  \subitem Ardoino UNO und/oder MEGA sind zwingend zu unterstützen
 \item Echtzeit loggen der Messwerte
 \item OpenSource (vermutlich MIT oder GNU Lizens)
 \item Konfigurierbarkeit der Diagramme
  \subitem Nutzer kann die diagramm darstellung weitestgehen beeinflussen:
    \subsubitem eine beliebige anzahl von diagrammen soll unterstützt werden
\end{itemize}

\section{Optionale Inhalte}
\begin{itemize}
 \item Interface bietet Möglichkeit zum Aufspielen der Arduino-Software
 \item Unterstützung mehrerer Chipsätze
 \item Diagramme können in separate Fenster ausgelagert werden
\end{itemize}

\section{Vorbereitung}
Zur Vorbereitung wurden unterschiedliche bestehende Hardware und Software 
Konzepte gesichtet und ausgewertet. Aus diesen Erkenntnissen wurden Schlüsse 
zur Verbesserung einiger Funktionalitäten sowie Ideen zur ``Nachahmung'' 
gezogen.

\section {Software}
DIe Software soll sich in ihrere Handhabung und ihrem Erscheinungsbild an 
Nicht-Informatiker richten. Die Konfiguration des Messaufbaus soll über 
grafisch ünterstützt werden, so das der Nutzer zu jeder Zeit weis welcher Pin 
des Arduinos, zum Messzeitpunkt, konfiguriert sein wird. 

Für die Log-Dateien soll eine Lokalisierung möglich sein. So soll der Nutzer zum 
Beispiel wählen können ob er lieber eine englische oder deutsche 
Zahlendarstellen haben will. Dies ist besonders von Interessen wenn es um die 
Weiter verarbeitung der Messwerte geht und diese in andere Programme importiert 
werden sollen. 

Das Interface soll aber nicht lokalisiert werden. Ich sehe hierfür keine 
Notwendigkeit.

\section {Arduino}
Die Arduino Software soll prinzipiell keine vim Nutzer definierte Logik 
ausfürhen. Ihre Aufgabe wird es sein auf die, über die bestehende 
USB-Verbindung empfangenen Befehle zu erkennen und zu verarbeiten. Diese 
Befehle können in dem Teil Befehlssatz eingesehen werden.
\subsection{Befehlssatz}
\begin{tabular}{l|l|c|l}
Befehl&Parameter& Richtung&Beschreibung\\\hline
setDPin&Id des digital Pins&IN&Konfiguriert einen digital Pins als entweder 
Input oder Output\\
setAPin&Id des analog Pins&IN&Konfiguriert einen analogen Pins als entweder 
Input oder Output\\
fetchDPin&Id des digital Pins&OUT&Gibt den aktuellen Zustand eines digital Pins 
wieder\\
fetchAPin&Id des analog Pins&OUT&Gibt den aktuellen Wert eines analogen Pins 
an\\
\end{tabular}

\subsubsection{Begriffsklärung}
\begin{itemize}
 \item[IN] von PC zu Arduino
 \item[OUT] von PC zu Arduino mit Rückgabe von Werten an PC
\end{itemize}

