\section{Grundlegende Ideen}
Die Bachelorarbeit soll zum Ziel haben ein Programm zu erarbeiten, mit dessen
Hilfe ein informatisch unversierter Nutzer Messungen in einer
elektrotechnischen Schaltung vornehmen kann ohne vorhandene
Programmierkenntnisse.\\
Der Titel könnte wie folgt lauten:
\textit{``Realisierung eines Datenloggers und Datenerfassungssystems mit Arduino 
UNO/Mega…\\
…zum Einsatz in wissenschaftlicher Experimentierumgebung\\
…''}

\section{Grundlagen / theoretischer Hintergrund / Motivation}
\begin{itemize}
 \item Was ist ein DatenLogger:
 \subitem Generelle Literatur zur Definition und Klärung wichtiger Terminologie
 \item Wie sieht die praktische Anwendung aus:
 \subitem Feld/Labor
 \subitem Kernkomponenten
 \subitem Campell Logger, und andere
 \item Grundlagen Arduino
 \subitem Warum, Was, Trend
 \subitem Specs
 \subitem \textit{ggf. Mikrocontroller allgemein}
 \item Case-Study
 \subitem Anwendung, Einsatz und Zweck von Datenloggern im Wissenschaftsbetrieb
 \subitem Erläuterung von beispielhaften Szenarien
 \subitem Welcher DatenLogger soll als Blaupause dienen?; CR100?
 \subitem Wie sieht die Programmierung aus (CRBasic vs. EdLog)
 \subitem Nachteile / Vorteile gegenüber Arduino
 \item Spezifikationen Arduino:
 \subitem Vorteile und Nachteile gegenüber DatenLogger
 \subitem Einschränkungen im Einsatz (Bit-Resolution, Anzahl der Kanäle, nur 
Single-Ended,…)
 \subitem 12-Bit A/D-Wandler ist Standard
 \subitem 16-Bit A/D von Adafruit ist möglich 
\end{itemize}


\section{Hauptinhalte}
\begin{itemize}
 \item Nutzerfreundliches Interface
 \item ``einfaches'' Handhabung der konfiguration
  \subitem Ardoino UNO und/oder MEGA sind zwingend zu unterstützen
 \item Echtzeitloggen der mehrerer Messwerte
 \item Laufzeit (ausgenommen laufende Messungen) Eingabe von 
Übertragungsfunktionen -> siehe LUA oder irgendeine 
Skriptsprache
 \item OpenSource (vermutlich MIT oder GNU Lizens)
 \item Konfigurierbarkeit der Diagramme
  \subitem Nutzer kann die diagramm darstellung weitestgehen beeinflussen:
    \subsubitem eine beliebige anzahl von diagrammen soll unterstützt werden
    \subsubitem scalierung der Diagramm soll möglich sein
\end{itemize}

\section{Optionale Inhalte}
\begin{itemize}
 \item Interface bietet Möglichkeit zum Aufspielen der Arduino-Software
 \item Unterstützung mehrerer Chipsätze (Arduino Uno | Arduino MEGA etc.)
 \item Diagramme können in separate Fenster ausgelagert werden
 \item Einfache ``Reaktionen'' sollen umgesetzt werden können
  \subitem z.B. Input A geht von \textbf{LOW} zu \textbf{HIGH} dann setze Output 
B
\end{itemize}

%\section{Vorbereitung}
%Zur Vorbereitung wurden unterschiedliche bestehende Hardware und Software
%Konzepte gesichtet und ausgewertet. Aus diesen Erkenntnissen wurden Schlüsse
%zur Verbesserung einiger Funktionalitäten sowie Ideen zur ``Nachahmung''
%gezogen.

\section {Software}
Die Software soll sich in ihrere Handhabung und ihrem Erscheinungsbild an
Nicht-Informatiker richten. Die Konfiguration des Messaufbaus soll über eine
grafische Oberfläche ünterstützt werden, so das der Nutzer zu jeder Zeit weis 
welcher Pin des Arduinos, zum Messzeitpunkt, konfiguriert sein wird.

Für die Log-Dateien soll eine Lokalisierung möglich sein. So soll der Nutzer zum
Beispiel wählen können ob er lieber eine englische oder deutsche
Zahlendarstellen haben will. Dies ist besonders von Interessen wenn es um die
Weiterverarbeitung der Messwerte geht und diese in andere Programme importiert
werden sollen.

Das Interface selbst soll nicht lokalisiert werden. Englisch als Sprache ist 
ausreichend.

\section {Arduino}
Die Arduino Software soll prinzipiell keine vim Nutzer definierte Logik
ausfürhen. Ihre Aufgabe wird es sein auf die, über die bestehende
USB-Verbindung empfangenen Befehle zu erkennen und zu verarbeiten. Diese
Befehle können in dem Teil Befehlssatz eingesehen werden.
\subsection{Befehlssatz}
\begin{tabular}{l|l|c|l}
\textbf{Befehl}&\textbf{Parameter}& 
\textbf{Richtung}&\textbf{Beschreibung}\\\hline
setDPin&Id des digital Pins&IN&Konfiguriert einen digital Pins als entweder
Input oder Output\\
setAPin&Id des analog Pins&IN&Konfiguriert einen analogen Pins als entweder
Input oder Output\\
fetchDPin&Id des digital Pins&OUT&Gibt den aktuellen Zustand eines digital Pins
wieder\\
fetchAPin&Id des analog Pins&OUT&Gibt den aktuellen Wert eines analogen Pins
an\\
readyArduino& - &IN&Prüft ob Arduino fertig initialisiert ist und Befehle 
entgegen nimmt\\
readyPC& - &OUT&Antwort auf \textit{readyArduino}. Bestätigt das der Arduino 
einsatzbereit ist\\
whoAreYou& - &IN&Fragt Arduino nach seinem Model\\
iAm& ModelId &OUT&Antwort auf \textit{whoAreYou}\\
\end{tabular}

\subsubsection{Begriffsklärung}
\begin{itemize}
 \item[IN] von PC zu Arduino
 \item[OUT] von PC zu Arduino mit Rückgabe von Werten an PC
\end{itemize}

