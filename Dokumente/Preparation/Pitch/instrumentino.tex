\subsection{Instrumentino und Controlino}
Istrumentino und die dazu gehörige Controlino Software für den Arduino wurden 
von Israel Joel Koenka, Jorge Sáiz, Peter C. Hauser entwickelt. Die Software 
verfolg den Ansatz einer Nutzer getriebenen Anpassung der Umgebung an eine 
bestimmte Aufgabe, wie z.B. die Messung von Werten aus einem MassFlowControler.
Hierzu wird ein durchaus überdurchschnittliches Informatikfachverständnis 
vorraus gesetzt.

\subsubsection{Vorteile}
\begin{itemize}
 \item eine Konfiguration kann als eigenständiges Programm auf Basis der 
Bibliothek gesehen werden und ist somit, weitest gehend, vor versehentlicher 
Änderung der Konfiguration geschütz
\end{itemize}

\subsubsection{Nachteile}
\begin{itemize}
 \item Der Installationsprozess gestaltet sich als umständlich und setzt zu 
weilen Nieschensoftware vorraus, die zusätzlich installiert werden muss.
 \item Eine Dokumentation ist nicht vorhanden -> Die Konfiguration eines 
Versuchsaufbaus gestaltet sich als umständlich
 \item Die Konfiguration der Software kann nur durch eine Person mit 
Informatik/Python Kenntnissen durchgeführt werden.
 \item Die Conrtolino Software verfügt über keine mitgelieferte Möglichkeit 
ohne größere Anpassung des Codes und schreiben eines Makefiles o.ä. für andere 
Arduino Boards aufgearbeitet und aufgespielt zu werden.
 \item Die Software selber leidet unter Bugs, die z.B. Grafiken nicht richtig 
oder garnicht laden
 \item Die Darstellungen mehrerer Signale wird nur gebündelt in einem Diagramm 
unterstützt. Es fehlt eine Option zur Erstellung mehrerer Diagramme
 \item Der CSV Logger beginnt bereits zum Start der Software mit der 
Logprozess, auch wenn garkein Arduino angeschlossen ist. -> Die Logdateien 
müssen aufgearbeitet und die Leerenzeilen entfernt werden.
\end{itemize}
