Die Arduino Plattform basiert auf der Diplomarbeit von Hernando Barragan und der darin behandelten und entwickelten Wiring-Plattform. Ziel der Diplomarbeit was die Entwicklung einer Plattform, welche das Lehren von Technoligie in verschiedensten Lernumgebungen zu ermöglichen bzw. zu vereinfachen\cite{hernandobarraganwiring2004}. Die Platformen, welche für Designer verfügbar waren, hatten häufig Elektrotechiker als Nutzer und setzten somit Grundkenntnisse in Elektrotechnik und Informatik vorraus. Barragan entwickelte den ersten Arduino vorgänger, dessen Layout auch in den neueren Arduino Boards erkennbar ist. Als Programmiersprache wurde C gewählt. Die Wahl war durch aus durch die Verbreitung der in den 70er Jahren entwickelten Sprache beeinflusst. Um den Nutzern, welche Barragan als Designer und somit Komilitonen seiner Universität dem Interaction Design Institute Ivrea ausmachte, den Einstieg in diese recht neue Form von Prototyping zu vereinfachen, bediente er sich bei der breits exsistenten Entwicklungsumgebung Processing\cite{caseyreasprocessing2001}. Auch der Ansatz eines geringen Einstiegsniveau für die IDE\todo{ACRO} wurde von Processing übernommen.

%was kam danach

%aktuelle geschichten

%aufbau platform
Die ersten Arduino Platformen basierten auf Atmel Microprozessoren. Bei dem Design der Boards wurde darauf geachtet, so viele Pins wie möglich zugänglich zu machen. Die Microprozessoren sind bei der Auslieferung bereits mit einem Bootloader bespielt und können bei Erhalt sofort mit einem Programm via der seriellen Schnittstelle programmiert werden. Da das Design der Boards offen zugänglich ist und die Entwicklung von ergänzender Hardware unterstützt wird, bietet die Platfrom optimale Voraussetzungen für Hobby DYI\todo{ARCO} Projekte und Protoypen. Beinahe jeder Aspekt der Platform kann nach den Wünschen der Nutzer geändert werden. So kann ein kundiger Nutzer den mit gelieferten Bootloader gegen einen potentiell schnelleren austauschen oder für die Plattform sogenannte Shields entwickeln, um weitere Funktionalitäten zu erlangen.

Die Verbeitung der Arudino Platform und das anhaltende Interesse an ihr\cite{google} befähigt sie zu einem hervoragenden preigünstigen Mittel für Prototypen und elektrotechnischen Aufbauen.
