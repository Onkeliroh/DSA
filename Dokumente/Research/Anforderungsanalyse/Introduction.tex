\chapter{Einleitung}
\section{Anlass}
Die Idee zu diesem Bachelorthema ist aus dem Arbeitsalltag am Max-Plack Institut
für Biogeochemie in Jena entstanden. Durch Gespräche mit Kollegen und beim
\gls{Brainstorming} mit meinem Vorgesetzten bin ich zu dem Entschluss gekommen, 
dass das angestrebte Thema eine sinnvolle Ergänzung im Arbeitsalltag und eine
Verbesserung der Arbeitsabläufe, sowohl im Institut als auch anders wo
darstellt.

\section{Zweck}
Zweck dieser Arbeit soll es sein, eine Software zu entwickeln die den Aufbau und
die Wartung von Versuchsaufbauten beschleunigt und vor allem vereinfacht. Bei meiner Recherche stellte sich heraus, dass manche der am Institut
verwendeten Systeme lediglich von einer geringen Anzahl Personen konfiguriert und gewartet werden können. Dieser Zustand ist keineswegs optimal und kann, z.B. durch das Fernbleiben eben dieser Personen, den betrieblichen Ablauf erheblich stören.
\section{Ziele}
Diese Arbeit hat zum Ziel eine Softwarelösung zu entwickeln und umzusetzen, die 
es einem elektrotechnisch geschulten und in dem Umgang mit Computern geübten 
Nutzer erlaubt einen elektronischen Schaltkreis zu überwachen, zu messen und zu 
steuern. Dem Nutzer soll eine Möglichkeit geboten werden, mittels eines eingängigen Interfaces, Messungen an elektrotechnischen Schaltungen vor zu nehmen. Darüber hinaus soll die Erstellung von Schaltvorgängen in einer elektrotechnischen Schaltung ermöglicht werden.
\section{Referenzen}
Bei Recherchen verfestigte sich mein Eindruck, dass eine solche Software durch aus Potential hat, Wissenschaftlern und \gls{Maker}{Makern} das Leben und ihre Arbeit zu vereinfachen. So ist das Interesse an erschwinglicher Hardware und handlicher 
Software zum Prototyping und dem Aufbau von Versuchen besonders durch das 
Aufkommen des Arduinos gewachsen. D.~K.~Fisher und P.~J.~Gould~\cite{ModernInstrumentation} beschrieben in ihrem Paper einen 
Versuchsaufbau mit Hilfe eines Arduino-Nachbaus zur Ermittlung von 
Feuchtigkeitswerten in und um Pflanzen herum. Ihr Aufbau und vorallem die 
Software verfolgen ein ähnliches Prinzip wie das von mir angestrebte. Der 
Arduino dient ihnen alls Datenerfassungsergrät und übermittelt alle Daten ohne 
weitere Verarbeitung direkt an entweder ein Speichermedium oder an einen 
Server. In diesem und einigen anderen Papers geht es häufig auch um eine finanzielle Sicht auf Experimente und die damit verbundene Durchführbarkeit. So beschreibt \citet{joshua_m._pearce_open-source_2014} in Kapitel 4 seines Buches, wie er eine Feuchtigkeitskammer für seine Forschungsarbeit anschaffen wollte. Er beschreibt, das die kommerziell verfügbaren Produkte außerhalb des Finanzrahmens waren und man somit auch eine \acrshort{DYI}-Lösung zurückgreifen mussten. Herz dieser Lösung war, aufgrund seines Preises und der beständigen Unterstützung durch Software-Bibliotheken Dritter, der Arduino.

Der Aspekt einer Datenerfassung und Auswertung hat bereits andere veranlasst 
eine entsprechende Software zu erstellen. Einer der jüngeren Ansätze stammt 
aus der Universität Basel~\cite{Instrumentino} und umfasst eine Software 
(Instrumentino) mit deren Hilfe extra angefertigte Arduino-Sketche (Controlino) 
und eine problembezogen konfigurierte Oberfläche zum Einsatz kommen. 
Nach eindringlicher Analyse der Software und einem Versuch eine eigene 
Testkonfiguration mit der Software, mit Hilfe des Entwicklers, zu erstellen, bin 
ihc zu dem Schluss gekommen das diese Software nicht als ``fertig'' angesehen 
werden kann und nicht einsatzbereit ist. Der Entwickler weißt in einer 
\href{https://github.com/yoelk/Instrumentino/blob/master/documents/Instrumentino\%20presentation.pptx}{Präsentation} auch darauf hin das der Konfigurationsprozess einen "`Administrator/Entwickler"' benötigt und es nicht vorgesehen ist diesen Schritt einem Endnutzer zu überlassen. 
Ein ausgereiftes Konzept, wenn auch etwas älter stammt von \citet{dedrick_inexpensive_2000} und beschreibt einen Datenlogger auf Mikrocontrollerbasis mit einer entsprechenden Software zum Auslesen der gesammelten Daten. Auch hier wurde Wert auf den finanziellen Rahmen, sowie die Weiterverarbeitung und Langzeiterfassung von Daten gelegt.